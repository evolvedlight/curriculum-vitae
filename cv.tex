%% start of file `template.tex'.
%% Copyright 2006-2015 Xavier Danaux (xdanaux@gmail.com).
%
% This work may be distributed and/or modified under the
% conditions of the LaTeX Project Public License version 1.3c,
% available at http://www.latex-project.org/lppl/.


\documentclass[11pt,a4paper,sans]{moderncv}        % possible options include font size ('10pt', '11pt' and '12pt'), paper size ('a4paper', 'letterpaper', 'a5paper', 'legalpaper', 'executivepaper' and 'landscape') and font family ('sans' and 'roman')

% moderncv themes
\moderncvstyle{classic}                             % style options are 'casual' (default), 'classic', 'banking', 'oldstyle' and 'fancy'
\moderncvcolor{purple}                               % color options 'black', 'blue' (default), 'burgundy', 'green', 'grey', 'orange', 'purple' and 'red'
%\renewcommand{\familydefault}{\sfdefault}         % to set the default font; use '\sfdefault' for the default sans serif font, '\rmdefault' for the default roman one, or any tex font name
%\nopagenumbers{}                                  % uncomment to suppress automatic page numbering for CVs longer than one page

% adjust the page margins
\usepackage[scale=0.78]{geometry}
%\setlength{\hintscolumnwidth}{3cm}                % if you want to change the width of the column with the dates
%\setlength{\makecvtitlenamewidth}{10cm}           % for the 'classic' style, if you want to force the width allocated to your name and avoid line breaks. be careful though, the length is normally calculated to avoid any overlap with your personal info; use this at your own typographical risks...

% personal data
\name{Stephen}{Brown}
\title{Curriculum Vitae}                               % optional, remove / comment the line if not wanted
\phone[mobile]{+41 786655435}                   % optional, remove / comment the line if not wanted; the optional "type" of the phone can be "mobile" (default), "fixed" or "fax"
\email{steve@brown.bg}                               % optional, remove / comment the line if not wanted
\homepage{brown.bg}                         % optional, remove / comment the line if not wanted
\social[github]{evolvedlight}                              % optional, remove / comment the line if not wanted
\extrainfo{C Permit}                 % optional, remove / comment the line if not wanted


% bibliography adjustements (only useful if you make citations in your resume, or print a list of publications using BibTeX)
%   to show numerical labels in the bibliography (default is to show no labels)
\makeatletter\renewcommand*{\bibliographyitemlabel}{\@biblabel{\arabic{enumiv}}}\makeatother
%   to redefine the bibliography heading string ("Publications")
%\renewcommand{\refname}{Articles}

% bibliography with mutiple entries
%\usepackage{multibib}
%\newcites{book,misc}{{Books},{Others}}
%----------------------------------------------------------------------------------
%            content
%----------------------------------------------------------------------------------
\begin{document}
%-----       resume       ---------------------------------------------------------
\makecvtitle

\section{Experience}
\cventry{2022--current}{Staff Software Engineer}{Vontobel}{Zurich}{}{First Staff Software Engineer at Vontobel in the Structured Products Engineering division, introducing the concept to drive communication and efficiency improvements\newline{}%
Highlights at Vontobel include:%
\begin{itemize}%
\item \textbf{Staff Software Engineer: } Introduced the concept of a Staff Software Engineer role at Vontobel to enhance cross-team communication and engineering efficiency.
\item \textbf{Public API Portal:} Developed and launched Vontobel's first external developer documentation portal, resulting in new profitable partnerships.
\item \textbf{Mifid/KID regulations revamp}: To demonstrate the benefits of a staff software engineer working across multiple teams, I rearchitected and rewrote key parts of a regulatory project spanning several teams and areas, reducing support efforts by ~1hr per day.
\end{itemize}}
Technologies used: \textbf{C\#, .Net Core, ASP.NET Core, MassTransit, Kafka, EF Core, Git, RabbitMQ, Kubernetes, Helm, Github}
\newline

\cventry{2019--2022}{Development Team Lead, Senior Software Engineer}{Gentwo}{Zurich}{}{Lead Developer at Gentwo. Full stack development from network cables to messaging systems.\newline{}%
As an early employee I took initiative to guide and improve the development team and software.
\newline
Highlights at Gentwo include:%
\begin{itemize}%
\item \textbf{Startup buildout: } As an early employee, I designed and built much of the infrastructure and systems used at Gentwo
\item \textbf{Azure/Managed services:} I consolidated the early technology choices into a single manageable platform, ensuring the whole team could focus on high value work rather than infrastructure maintenance
\item \textbf{Developer onboarding and recruitment}: I setup and organised our recruitment programs to ensure we could build out a high quality team
\item \textbf{Team organisation and agility}: As we grew from a single team into two, I helped both teams transition to a stable organised development process
\end{itemize}}
Technologies used: \textbf{C\#, .Net Core, ASP.NET Core, EF Core, Git, RabbitMQ, Vue.js, Kubernetes, Helm, Gitlab}
\newline

\cventry{2017--2019}{Senior Software Engineer Deritrade}{Vontobel}{Zurich}{}{Software Engineer in both the frontend and backend teams within Structured Products. \newline{}%
Project highlights at Vontobel include:%
\begin{itemize}%
\item \textbf{RabbitMQ: } Led the migration from MSMQ to RabbitMQ, improving scalability and enabling .Net Core adoption. This initiative was later adopted by other teams.
\item \textbf{Kubernetes/Containers:} Spearheaded the adoption of Linux Containers and OpenShift (Kubernetes) for the Structured Products department, standardizing architecture, reducing support costs, and improving scalability. Coordinated cross-team efforts and resolved technical debt.
\item \textbf{Support Improvements}: Identified and resolved inefficiencies in the Deritrade platform, reducing daily automated issues by 50% through business, technical, and process improvements.
\end{itemize}}
Technologies used: \textbf{C\#, .Net Core, ASP.NET Core, EF Core, Git, REST, RabbitMQ, MSMQ, NServiceBus, OpenShift, Docker, Vue.js, gRPC, Swagger}
\newline

\cventry{2015--2017}{Senior C\# programmer}{Quartal}{Zurich}{}{Team Lead, responsible for 6 developers. Worked with management and colleagues to improve efficiency of the team, transforming the team from one that often failed to deliver into a highly performing team.\newline{}%
Project highlights at Quartal include:%
\begin{itemize}%
\item \textbf{Release Automation:} To allow us to release faster and with fewer errors, I took the existing release process and automated most of it into a single click. This replaced the highly manual process that required several team members previously;
\end{itemize}}
Technologies used: \textbf{C\#, SQL/Oracle, Git, WCF, IoC (AutoFac), Teamcity, Visual Studio}
\newline

\cventry{2011--2015}{Programmer (Assistant Vice President)}{Bank of America}{London}{}{Engineer on a critical trading system at Bank of America for Equities. I worked on all parts of the application, spanning from the C\# frontend to the Java backend and release scripts. Responsibilities also included interacting with Markitwire (an automated inter-bank trade settlement system), performing regulatory checks, and communicating with downstream systems. \newline{}%
Project highlights in Bank of America include:%
\begin{itemize}%
\item \textbf{Hot-Hot servers:} To increase availability and speed of our production systems, I implemented a DNS and HTTP based hot-hot load balanced system, one of the first in the bank.
\item \textbf{REST/Swagger:} To allow other teams to benefit from the high performance system we had developed, we opened up our internal APIs using a REST/HTTP interface. I developed a single consistent JSON based REST api which was easy to use and trivial to develop against. This was automatically documented with Swagger.
\end{itemize}}
Technologies used: \textbf{C\#, Python, SQL/Oracle, Ansible, Haproxy, Git, Swagger, REST, Teamcity, NuGet, Kerberos}
\newline

\cventry{Summer 2010}{Google Summer of Code Student}{Google/OpenStreetMap}{UK}{}{I was accepted into Google Summer of Code to work on an open source Android application called OpenSatNav. \newline{}%
Project highlights while working for Google/OpenStreetMap include:%
\begin{itemize}%
\item Significant development on an open source Android application
\item Collaborated with team members and translators around the world
\item Took on responsibility for managing release schedule
\end{itemize}}
Technologies used: \textbf{Java, Android, SVN, OpenStreetMap, Google Play Store, XML, GPS}

\section{Education}
\cventry{2007--2011}{Master of Engineering in Computer Science}{University of Warwick}{}{\textit{2:1}}{}  % arguments 3 to 6 can be left empty

\section{Languages}
\cvitemwithcomment{English}{Native}{}
\cvitemwithcomment{German}{B2}{With B2 Certificate}
\cvitemwithcomment{Swiss German}{Basic}{}
\cvitemwithcomment{Bulgarian}{Intermediate}{ongoing lessons}

\section{Other interests}
I'm a keen whitewater kayaker, and enjoy skiing, paragliding, swimming, and running. I contribute to several open source projects in \textbf{C\#} and have a small interest in Flutter/Dart.

\clearpage
\end{document}


%% end of file `template.tex'.
